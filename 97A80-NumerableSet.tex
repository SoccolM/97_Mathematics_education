\documentclass[12pt]{article}
\usepackage{pmmeta}
\pmcanonicalname{NumerableSet}
\pmcreated{2013-03-22 16:01:32}
\pmmodified{2013-03-22 16:01:32}
\pmowner{juanman}{12619}
\pmmodifier{juanman}{12619}
\pmtitle{numerable set}
\pmrecord{11}{38066}
\pmprivacy{1}
\pmauthor{juanman}{12619}
\pmtype{Definition}
\pmcomment{trigger rebuild}
\pmclassification{msc}{97A80}
%\pmkeywords{Analysis}
\pmrelated{Calculus}
\pmrelated{TopicsOnCalculus}
\pmrelated{Denumerable}
\pmrelated{Countable}
\pmdefines{enumeration}
\pmdefines{enumerable}

\endmetadata

% this is the default PlanetMath preamble.  as your knowledge
% of TeX increases, you will probably want to edit this, but
% it should be fine as is for beginners.

% almost certainly you want these
\usepackage{amssymb}
\usepackage{amsmath}
\usepackage{amsfonts}

% used for TeXing text within eps files
%\usepackage{psfrag}
% need this for including graphics (\includegraphics)
%\usepackage{graphicx}
% for neatly defining theorems and propositions
%\usepackage{amsthm}
% making logically defined graphics
%%%\usepackage{xypic}

% there are many more packages, add them here as you need them

% define commands here

\begin{document}
Let $X$ be a set.  An \emph{enumeration} on $X$ is a surjection from the set of natural numbers $\mathbb{N}$ to $X$.

A set $X$ is called \emph{numerable} if there is a bijective enumeration on $X$.

It is easy to show that $\mathbb{Z}$ and $\mathbb{Q}$ are numerable.

It is a standard fact that $\mathbb{R}$ is not numerable. For, if we suppose that the numbers [0,1] were countable, we can arrange them in a list (given by the supposed bijection).

Representing them in a binary form, it is not hard to construct an element in [0,1], which is not in the list.

This contradiction implies that [0,1]$\subset\mathbb{R}$ is not numerable. 

\textbf{Remark}.  If the enumeration $\mathbb{N}\to X$ is furthermore a computable function, then we say that $X$ is \emph{enumerable}.  There exists numerable sets that are not enumerable.
%%%%%
%%%%%
\end{document}
