\documentclass[12pt]{article}
\usepackage{pmmeta}
\pmcanonicalname{AbsoluteValueInequalities}
\pmcreated{2013-03-22 15:59:20}
\pmmodified{2013-03-22 15:59:20}
\pmowner{Wkbj79}{1863}
\pmmodifier{Wkbj79}{1863}
\pmtitle{absolute value inequalities}
\pmrecord{7}{38009}
\pmprivacy{1}
\pmauthor{Wkbj79}{1863}
\pmtype{Definition}
\pmcomment{trigger rebuild}
\pmclassification{msc}{97D40}
\pmrelated{InequalityWithAbsoluteValues}
\pmdefines{greator less thand}

% this is the default PlanetMath preamble.  as your knowledge
% of TeX increases, you will probably want to edit this, but
% it should be fine as is for beginners.

% almost certainly you want these
\usepackage{amssymb}
\usepackage{amsmath}
\usepackage{amsfonts}

% used for TeXing text within eps files
%\usepackage{psfrag}
% need this for including graphics (\includegraphics)
%\usepackage{graphicx}
% for neatly defining theorems and propositions
%\usepackage{amsthm}
% making logically defined graphics
%%%\usepackage{xypic}

% there are many more packages, add them here as you need them

% define commands here

\begin{document}
Let $a,b,c \in \mathbb{R}$ and $f(x) \in \mathbb{R}[x]$.  There is a mnemonic device that is useful for solving inequalities of the following forms:

\begin{center}
$\begin{array}{ccc}
a|f(x)|+b \le c & \,\, & c \ge a|f(x)|+b \\
a|f(x)|+b < c & \,\, & c > a|f(x)|+b \\
a|f(x)|+b \ge c & \,\, & c \le a|f(x)|+b \\
a|f(x)|+b > c & \,\, & c < a|f(x)|+b \end{array}$
\end{center}

Before using the mnemonic device, the expression $|f(x)|$ should be \PMlinkescapetext{isolated} and on the left hand \PMlinkescapetext{side} of the inequality.  Once this is accomplished, the absolute value must be dealt with:  One statement should look \PMlinkescapetext{similar} to the previous one, the only \PMlinkescapetext{difference} being that the absolute value \PMlinkescapetext{signs} are dropped.  The other statement should also have the absolute value \PMlinkescapetext{signs} dropped, but the inequality needs reversed and the number (on the \PMlinkescapetext{right}) needs to be negated.

The two statements as described above should be \PMlinkescapetext{connected} using either $\operatorname{or}$ or $\operatorname{and}$.  The mnemonic that aids in remembering which one to use is {\sl greator less thand\/}.  That is, when the inequality before splitting up has $>$ or $\ge$, the connector $\operatorname{or}$ should be used; when the inequality before splitting up has $<$ or $\le$, the connector $\operatorname{and}$ should be used.

Here is an example:

\begin{center}
$\begin{array}{rl}
8 & > 3+|2x-7| \\
5 & > |2x-7| \\
|2x-7| & < 5 \end{array}$
\end{center}

Since the inequality is $<$, $\operatorname{and}$ should be used.

\begin{center}
$\begin{array}{rlcrl}
2x-7 & < 5 & \, \operatorname{and} \, & 2x-7 & > -5 \\
2x & < 12 & \, \operatorname{and} \, & 2x & > 2 \\
x & < 6 & \, \operatorname{and} \, & x & > 1 \end{array}$
\end{center}

$$1<x<6$$

I would like to thank Mrs. Sue Millikin, who taught me absolute value inequalities in this manner.
%%%%%
%%%%%
\end{document}
