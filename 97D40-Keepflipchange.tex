\documentclass[12pt]{article}
\usepackage{pmmeta}
\pmcanonicalname{Keepflipchange}
\pmcreated{2013-03-22 15:59:52}
\pmmodified{2013-03-22 15:59:52}
\pmowner{Wkbj79}{1863}
\pmmodifier{Wkbj79}{1863}
\pmtitle{keep-flip-change}
\pmrecord{4}{38023}
\pmprivacy{1}
\pmauthor{Wkbj79}{1863}
\pmtype{Definition}
\pmcomment{trigger rebuild}
\pmclassification{msc}{97D40}
\pmsynonym{KFC}{Keepflipchange}
\pmsynonym{K.F.C.}{Keepflipchange}

% this is the default PlanetMath preamble.  as your knowledge
% of TeX increases, you will probably want to edit this, but
% it should be fine as is for beginners.

% almost certainly you want these
\usepackage{amssymb}
\usepackage{amsmath}
\usepackage{amsfonts}

% used for TeXing text within eps files
%\usepackage{psfrag}
% need this for including graphics (\includegraphics)
%\usepackage{graphicx}
% for neatly defining theorems and propositions
%\usepackage{amsthm}
% making logically defined graphics
%%%\usepackage{xypic}

% there are many more packages, add them here as you need them

% define commands here

\begin{document}
Let $a,b,c,d \in \mathbb{Z}-\{0\}$.  A mnemonic device for calculating

$$\frac{a}{b} \div \frac{c}{d}$$

is {\sl KFC}.  This stands for "keep-flip-change".  This means that you keep the first fraction as it is, flip the second fraction, and change the problem to a multiplication problem.  Thus, the expression above becomes

$$\frac{a}{b} \cdot \frac{d}{c}.$$

This yields that

$$\frac{a}{b} \div \frac{c}{d}=\frac{a}{b} \cdot \frac{d}{c}=\frac{ad}{bc}.$$
%%%%%
%%%%%
\end{document}
