\documentclass[12pt]{article}
\usepackage{pmmeta}
\pmcanonicalname{StrangeRoot}
\pmcreated{2013-03-22 17:55:53}
\pmmodified{2013-03-22 17:55:53}
\pmowner{pahio}{2872}
\pmmodifier{pahio}{2872}
\pmtitle{strange root}
\pmrecord{8}{40426}
\pmprivacy{1}
\pmauthor{pahio}{2872}
\pmtype{Definition}
\pmcomment{trigger rebuild}
\pmclassification{msc}{97D99}
\pmclassification{msc}{26A09}
\pmsynonym{wrong root}{StrangeRoot}
\pmsynonym{extraneous root}{StrangeRoot}
\pmrelated{QuadraticFormula}
\pmrelated{LogicalOr}
\pmrelated{SquaringConditionForSquareRootInequality}

\endmetadata

% this is the default PlanetMath preamble.  as your knowledge
% of TeX increases, you will probably want to edit this, but
% it should be fine as is for beginners.

% almost certainly you want these
\usepackage{amssymb}
\usepackage{amsmath}
\usepackage{amsfonts}

% used for TeXing text within eps files
%\usepackage{psfrag}
% need this for including graphics (\includegraphics)
%\usepackage{graphicx}
% for neatly defining theorems and propositions
 \usepackage{amsthm}
% making logically defined graphics
%%%\usepackage{xypic}

% there are many more packages, add them here as you need them

% define commands here

\theoremstyle{definition}
\newtheorem*{thmplain}{Theorem}

\begin{document}
\PMlinkescapeword{root} \PMlinkescapeword{roots}

In solving certain \PMlinkescapetext{types} of equations, one may obtain besides the proper (\PMlinkescapetext{right}) \PMlinkname{roots}{Equation} also some {\em strange roots} which do not satisfy the original equation.  Such a thing can happen especially when one has in some stage squared both sides of the treated equation; in this situation one must check all ``roots'' by substituting them to the original equation.\\

\textbf{Example.}\, $$x-\sqrt{x} = 12$$
$$x-12 = \sqrt{x}$$
$$(x-12)^2 = (\sqrt{x})^2$$
$$x^2-24x+144 = x$$
$$x^2-25x+144 = 0$$
$$x = \frac{25\pm\sqrt{25^2-4\cdot144}}{2} = \frac{25\pm7}{2}$$
$$x = 16 \quad \lor \quad x = 9$$
Substituting these values of $x$ into the left side of the original equation yields
$$16-4 = 12, \quad 9-3 = 6.$$
Thus, only\, $x = 16$\, is valid,\, $x = 9$\, is a strange root.  (How\, $x = 9$\, is related to the solved equation, is explained by that it may be written\, $(\sqrt{x})^2-\sqrt{x}-12 = 0$, from which one would obtain via the quadratic formula that\, $\sqrt{x} = \frac{1\pm7}{2}$,\, i.e.\, $\sqrt{x} = 4$\, or\, $\sqrt{x} = -3$.\, The latter corresponds the value\, $x = 9$,\, but it were relevant to the original equation only if we would allow negative values for square roots of positive numbers; the \PMlinkescapetext{current} practice excludes them.)\\


The general explanation of strange roots when squaring an equation is, that the two equations
$$a = b,$$
$$a^2 = b^2$$
are not \PMlinkname{equivalent}{Equivalent3} (but the equations\, $a = \pm b$\, and\, $a^2 = b^2$\, would be such ones).



%%%%%
%%%%%
\end{document}
