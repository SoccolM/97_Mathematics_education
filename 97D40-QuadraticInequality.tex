\documentclass[12pt]{article}
\usepackage{pmmeta}
\pmcanonicalname{QuadraticInequality}
\pmcreated{2013-03-22 15:23:48}
\pmmodified{2013-03-22 15:23:48}
\pmowner{pahio}{2872}
\pmmodifier{pahio}{2872}
\pmtitle{quadratic inequality}
\pmrecord{12}{37233}
\pmprivacy{1}
\pmauthor{pahio}{2872}
\pmtype{Topic}
\pmcomment{trigger rebuild}
\pmclassification{msc}{97D40}
\pmclassification{msc}{26-00}
\pmclassification{msc}{12D99}
\pmrelated{QuadraticFormula}
\pmrelated{SolvingCertainPolynomialInequalities}
\pmrelated{TangentOfConicSection}
\pmrelated{IndexOfInequalities}

% this is the default PlanetMath preamble.  as your knowledge
% of TeX increases, you will probably want to edit this, but
% it should be fine as is for beginners.

% almost certainly you want these
\usepackage{amssymb}
\usepackage{amsmath}
\usepackage{amsfonts}

% used for TeXing text within eps files
%\usepackage{psfrag}
% need this for including graphics (\includegraphics)
\usepackage{graphicx}
% for neatly defining theorems and propositions
 \usepackage{amsthm}
% making logically defined graphics
%%%\usepackage{xypic}

% there are many more packages, add them here as you need them

% define commands here

\theoremstyle{definition}
\newtheorem*{thmplain}{Theorem}
\begin{document}
The \PMlinkescapetext{{\em normal form}} of a {\em quadratic inequality} is
\begin{align}
   ax^2\!+\!bx\!+\!c \;<\; 0
\end{align}
or 
\begin{align}
   ax^2\!+\!bx\!+\!c \;>\; 0
\end{align}
where $a$, $b$ and $c$ are known real numbers and\, $a \neq 0$.

Solving such an inequality, i.e. determining all real values of $x$ which satisfy it, is based on the fact that the graph of the quadratic polynomial function\, $x\mapsto ax^2\!+\!bx\!+\!c$\, is the parabola
    $$y \;=\; ax^2\!+\!bx\!+\!c,$$
opening upwards if\, $a > 0$\, and downwards if\, $a < 0$.

For obtaining the solution we first have to determine the real zeroes of the polynomial $ax^2\!+\!bx\!+\!c$, i.e. solve the \PMlinkname{quadratic equation}{QuadraticFormula}\, 
$ax^2\!+\!bx\!+\!c = 0$.
\begin{itemize}
 \item If there is two distinct real zeroes $x_1$ and $x_2$ (say\, $x_1 < x_2$),\, then the parabola intersects the $x$-axis in these points.\, In the case\, $a > 0$\, the parabola opens upwards and thus\, $y < 0$\, in the interval\, $(x_1,\,x_2)$, but\, $y > 0$\, outside this interval.\, I.e., for positive $a$,  the solution of (1) is
      $$x_1 \;<\; x \;<\; x_2$$
and the solution of (2) is
      $$x \;<\; x_1\,\,\,\mbox{ or }\,\,\,x \;>\; x_2$$ 
(note that the latter solution-domain consists of two distinct portions of the $x$-axis and therefore must be expressed with two separate inequalities, not with a double inequality as the former).\, For negative $a$ we must swap those solutions for (1) and (2).

\begin{figure}[!htb]
\begin{center}
\includegraphics{parabola.1.eps}
\end{center}
\caption{Solving for $ax^2\!+\!bx\!+\!c < 0$ when $a > 0$ and the quadratic has two distinct roots}
\end{figure}

 \item If there is only one real zero of the polynomial (we may say that\, $x_2 = x_1$), the parabola has $x$-axis as the \PMlinkname{tangent}{TangentLine} in its apex.\, For positive $a$ the other points of parabola are above the $x$-axis, i.e. they have\, $y > 0$\, always\, but\, $y < 0$\, never.\, So, (1) has no solutions, but (2) is true for all\, $x \neq x_1$ (i.e.\, $x < x_1$\, or\, $x > x_1$).\, For the case of negative $a$ we again must change those solutions for (1) and (2).

\begin{figure}[!htb]
\begin{center}
\includegraphics{parabola.2.eps}
\end{center}
\caption{Solving for $ax^2\!+\!bx\!+\!c > 0$ when $a > 0$ and the quadratic has only one root}
\end{figure}

 \item There can still appear the possibility that the polynomial has no real zeroes (the roots of the equation are imaginary).\, Now the parabola does not intersect or touch the $x$-axis, but is totally above the axis when $a$ is positive ($y > 0$\, always) and totally below the axis when $a$ is negative 
($y < 0$\, always).\, Thus we get no solutions at all (the inequality is impossible) or all real numbers $x$ as solutions, according to what the inequality (1) or (2) demands.

\begin{figure}[!htb]
\begin{center}
\includegraphics{parabola.3.eps}
\end{center}
\caption{$ax^2\!+\!bx\!+\!c > 0$ for all $x$ when $a > 0$ and the quadratic has no roots}
\end{figure}

\end{itemize}
%%%%%
%%%%%
\end{document}
