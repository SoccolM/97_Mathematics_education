\documentclass[12pt]{article}
\usepackage{pmmeta}
\pmcanonicalname{ProportionalityOfNumbers}
\pmcreated{2014-02-23 21:26:01}
\pmmodified{2014-02-23 21:26:01}
\pmowner{pahio}{2872}
\pmmodifier{pahio}{2872}
\pmtitle{proportionality of numbers}
\pmrecord{11}{42093}
\pmprivacy{1}
\pmauthor{pahio}{2872}
\pmtype{Definition}
\pmcomment{trigger rebuild}
\pmclassification{msc}{97U99}
\pmclassification{msc}{12D99}
\pmsynonym{proportionality}{ProportionalityOfNumbers}
\pmrelated{Variation}
\pmrelated{KalleVaisala}
\pmdefines{proportional}
\pmdefines{directly proportional}
\pmdefines{inversely proportional}

\endmetadata

% this is the default PlanetMath preamble.  as your knowledge
% of TeX increases, you will probably want to edit this, but
% it should be fine as is for beginners.

% almost certainly you want these
\usepackage{amssymb}
\usepackage{amsmath}
\usepackage{amsfonts}

% used for TeXing text within eps files
%\usepackage{psfrag}
% need this for including graphics (\includegraphics)
%\usepackage{graphicx}
% for neatly defining theorems and propositions
 \usepackage{amsthm}
% making logically defined graphics
%%%\usepackage{xypic}

% there are many more packages, add them here as you need them

% define commands here

\theoremstyle{definition}
\newtheorem*{thmplain}{Theorem}

\begin{document}
\PMlinkescapeword{members}

The nonzero numbers $a_1,\,a_2,\,\ldots,\,a_n$ are (\emph{directly}) \emph{proportional} to the nonzero numbers 
$b_1,\,b_2,\,\ldots,\,b_n$ if
\begin{align}
a_1\!:\!a_2\!:\ldots:\!a_n \;=\; b_1\!:\!b_2\!:\ldots:\!b_n,
\end{align}
which special notation means the simultaneous validity of the proportion equations
\begin{align}
a_1\!:\!a_2 \;=\; b_1\!:\!b_2, 
\;\quad a_2\!:\!a_3 \;=\; b_2\!:\!b_3,\;\quad \ldots, \;\quad 
a_{n-1}\!:\!a_n \;=\; b_{n-1}\!:\!b_n.
\end{align}

It follows however that 
\begin{align}
a_i\!:\!a_j \;=\; b_i\!:\!b_j \quad \mbox{for all}\;\; i,\,j.
\end{align}
In fact, if one multiplies the left hand sides of e.g. two first equations (2) and similarly their right hand sides, then one obtains\, $a_1\!:\!a_3 \;=\; b_1\!:\!b_3$.

Swapping the middle members of the proportions (2), which by the \PMlinkname{parent entry}{ProportionEquation} is allowable, one gets the system of equations 
\begin{align}
\frac{a_1}{b_1} \;=\; \frac{a_2}{b_2} \;=\; \ldots \;=\; \frac{a_n}{b_n}
\end{align}
which is \PMlinkname{equivalent}{Equivalent3} with (1) and (2).\\

The numbers $a_1,\,a_2,\,\ldots,\,a_n$ are \emph{inversely proportional} to the numbers $b_1,\,b_2,\,\ldots,\,b_n$ if
$$a_1\!:\!a_2\!:\ldots:\!a_n \;=\; \frac{1}{b_1}\!:\!\frac{1}{b_2}\!:\ldots:\!\frac{1}{b_n}.$$
Then we have
$$a_i\!:\!a_j \;=\; b_j\!:\!b_i \quad \mbox{for all}\;\; i,\,j.$$\\


\textbf{Note.}\, The notation \,$a_1\!:\!a_2\!:\ldots:\!a_n$\, expressing the ``ratio of several numbers'' is, of course, \PMlinkescapetext{not a number}, but it behaves as the ratio (= the quotient) of two numbers in the sense that all of its members $a_i$ may be multiplied by a nonzero number without injuring the validity of (1).\\

\textbf{Example.}\, Let\, $a\!:\!b = 2\!:\!3$\, and\, $b\!:\!c = 4\!:\!5$.\, Determine the least positive integers to which the numbers $a,\,b,\,c$ are\, a) directly,\, b) inversely proportional.\\
a) The least common multiple of 3 and 4, the members corresponding the members $b$ in the given proportions, is 12.\, Thus we must multiply the right hand sides of these proportions respectively by\, $\frac{12}{3} = 4$\, and\, $\frac{12}{4} = 3$:
$$a\!:\!b \;=\; 2\!:\!3 \;=\; 8\!:\!12, \quad b\!:\!c \;=\; 4\!:\!5 \;=\; 12\!:\!15.$$
Accordingly,
$$a\!:\!b\!:\!c \;=\; 8\!:\!12\!:\!15.$$
b) We may write
$$a\!:\!b \;=\; \frac{1}{3}\!:\!\frac{1}{2} \;=\; \frac{1}{15}\!:\!\frac{1}{10}, \quad 
  b\!:\!c \;=\; \frac{1}{5}\!:\!\frac{1}{4} \;=\; \frac{1}{10}\!:\!\frac{1}{8},$$
where the denominators of the right hand sides have been multiplied by 5 and 2, respectively.\, Consequently,
$$a\!:\!b\!:\!c \;=\; \frac{1}{15}\!:\!\frac{1}{10}\!:\!\frac{1}{8},$$
i.e. the required integers are 15, 10, 8.


\begin{thebibliography}{8}
\bibitem{VG}{\sc K. V\"ais\"al\"a}: {\em Geometria}.\, Reprint of the tenth edition.\, Werner S\"oderstr\"om Osakeyhti\"o, Porvoo \& Helsinki (1971).
\end{thebibliography}

%%%%%
%%%%%
\end{document}
