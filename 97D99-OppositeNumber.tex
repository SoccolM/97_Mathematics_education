\documentclass[12pt]{article}
\usepackage{pmmeta}
\pmcanonicalname{OppositeNumber}
\pmcreated{2013-03-22 15:03:25}
\pmmodified{2013-03-22 15:03:25}
\pmowner{pahio}{2872}
\pmmodifier{pahio}{2872}
\pmtitle{opposite number}
\pmrecord{10}{36775}
\pmprivacy{1}
\pmauthor{pahio}{2872}
\pmtype{Definition}
\pmcomment{trigger rebuild}
\pmclassification{msc}{97D99}
\pmclassification{msc}{12D99}
\pmsynonym{negative [as a noun]}{OppositeNumber}
%\pmkeywords{addition}
\pmrelated{Ring}
\pmrelated{OppositePolynomial}
\pmrelated{ConditionOfOrthogonality}
\pmrelated{Automorphism4}
\pmrelated{ProductOfNegativeNumbers}
\pmrelated{PlusSign}

\endmetadata

% this is the default PlanetMath preamble.  as your knowledge
% of TeX increases, you will probably want to edit this, but
% it should be fine as is for beginners.

% almost certainly you want these
\usepackage{amssymb}
\usepackage{amsmath}
\usepackage{amsfonts}

% used for TeXing text within eps files
%\usepackage{psfrag}
% need this for including graphics (\includegraphics)
%\usepackage{graphicx}
% for neatly defining theorems and propositions
%\usepackage{amsthm}
% making logically defined graphics
%%%\usepackage{xypic}

% there are many more packages, add them here as you need them

% define commands here

\begin{document}
The {\em opposite number} of a number $a$ is such a number\, $-a$\, that
              $$-a\!+\!a \;=\; 0.$$

Some properties:
\begin{itemize}
 \item $-a \;=\; (-1)\!\cdot\!a$
 \item $-0 \;=\; 0$
 \item $-(-a) \;=\; a$
 \item $-(a\!+\!b) \;=\; (-a)\!+\!(-b)$
 \item $-(a\!\cdot\!b) \;=\; a\!\cdot\!(-b) \;=\; (-a)\!\cdot\!b$
 \item $-(a\!-\!b) \;=\; b\!-\!a$
 \item $-\sum_{j = 1}^n a_j \;=\; \sum_{j = 1}^n (-a_j)$
 \item $-\int_a^b f(x)\,dx \;=\; \int_b^a f(x)\,dx$
\end{itemize}
Exactly similar properties (except the last) are valid in every ring.\, The fifth property implies the

\textbf{Corollary.}\, If one changes the sign of one factor of a ring product, then the sign of the whole product changes.

%%%%%
%%%%%
\end{document}
