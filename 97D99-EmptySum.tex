\documentclass[12pt]{article}
\usepackage{pmmeta}
\pmcanonicalname{EmptySum}
\pmcreated{2013-03-22 18:40:57}
\pmmodified{2013-03-22 18:40:57}
\pmowner{pahio}{2872}
\pmmodifier{pahio}{2872}
\pmtitle{empty sum}
\pmrecord{5}{41433}
\pmprivacy{1}
\pmauthor{pahio}{2872}
\pmtype{Topic}
\pmcomment{trigger rebuild}
\pmclassification{msc}{97D99}
\pmclassification{msc}{05A19}
\pmclassification{msc}{00A05}
\pmrelated{EmptyProduct}
\pmrelated{EmptySet}
\pmrelated{AddingAndRemovingParenthesesInSeries}

% this is the default PlanetMath preamble.  as your knowledge
% of TeX increases, you will probably want to edit this, but
% it should be fine as is for beginners.

% almost certainly you want these
\usepackage{amssymb}
\usepackage{amsmath}
\usepackage{amsfonts}

% used for TeXing text within eps files
%\usepackage{psfrag}
% need this for including graphics (\includegraphics)
%\usepackage{graphicx}
% for neatly defining theorems and propositions
 \usepackage{amsthm}
% making logically defined graphics
%%%\usepackage{xypic}

% there are many more packages, add them here as you need them

% define commands here

\theoremstyle{definition}
\newtheorem*{thmplain}{Theorem}

\begin{document}
The {\em empty sum} is such a borderline case of sum where the number of the addends is zero, i.e. the set of the addends is an empty set.

\begin{itemize}

\item One may think that the zeroth multiple $0a$ of a ring element $a$ is the empty sum; it can spring up by adding in the ring two multiples whose integer coefficients are opposite numbers:
$$(-n)a\!+\!na \,=\, (-n\!+\!n)a = 0a$$
This empty sum equals the additive identity 0 of the ring, since the multiple $(-n)a$ is defined to be
$$\underbrace{(-a)\!+\!(-a)\!+\ldots+\!(-a)}_{n\; \mathrm{copies}}$$

\item In using the \PMlinkname{sigma notation}{Summing} 
\begin{align}
\sum_{i=m}^nf(i)
\end{align}
one sometimes sees a case 
\begin{align}
\sum_{i=m}^{m-1}f(i).
\end{align}
It must be an empty sum, because in
\begin{align}
\sum_{i=m}^mf(i)
\end{align}
the number of addends is clearly one and therefore in (2) the number is zero.\, Thus the value of (2) may be defined to be 0.
\end{itemize}



\textbf{Note.}\, The sum (1) is not defined when $n$ is less than $m\!-\!1$, but if one would want that the usual rule
\begin{align}
\sum_{i=m}^nf(i)+\sum_{i=n+1}^kf(i) \;=\; \sum_{i=m}^kf(i)
\end{align}
would be true also in such a cases, then one has to define
$$\sum_{i=m}^nf(i) \;=\; -\sum_{i=n+1}^{m-1}f(i) \qquad\qquad(n < m\!-\!1),$$
because by (4) one could calculate
$$0 \,=\, -\sum_{i=n+1}^{m-1}f(i)+\sum_{i=n+1}^{m-1}f(i) \,=\,\sum_{i=m}^nf(i)+\sum_{i=n+1}^{m-1}f(i) 
\,=\, \sum_{i=m}^{m-1}f(i).$$


%%%%%
%%%%%
\end{document}
