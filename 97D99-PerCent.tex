\documentclass[12pt]{article}
\usepackage{pmmeta}
\pmcanonicalname{PerCent}
\pmcreated{2013-03-22 17:32:43}
\pmmodified{2013-03-22 17:32:43}
\pmowner{CWoo}{3771}
\pmmodifier{CWoo}{3771}
\pmtitle{per cent}
\pmrecord{7}{39947}
\pmprivacy{1}
\pmauthor{CWoo}{3771}
\pmtype{Definition}
\pmcomment{trigger rebuild}
\pmclassification{msc}{97D99}
\pmclassification{msc}{00A69}
\pmsynonym{percent}{PerCent}
\pmdefines{percentage point}
\pmdefines{per cent number}
\pmdefines{base value}
\pmdefines{per cent value}

\endmetadata

% this is the default PlanetMath preamble.  as your knowledge
% of TeX increases, you will probably want to edit this, but
% it should be fine as is for beginners.

% almost certainly you want these
\usepackage{amssymb}
\usepackage{amsmath}
\usepackage{amsfonts}

% used for TeXing text within eps files
%\usepackage{psfrag}
% need this for including graphics (\includegraphics)
%\usepackage{graphicx}
% for neatly defining theorems and propositions
 \usepackage{amsthm}
% making logically defined graphics
%%%\usepackage{xypic}

% there are many more packages, add them here as you need them

% define commands here

\theoremstyle{definition}
\newtheorem*{thmplain}{Theorem}

\begin{document}
\PMlinkescapeword{mean}

The \PMlinkescapetext{word} {\em per cent} may be in general interpreted to mean a `hundredth'.  So e.g. 5 per cent is `5 hundredths', i.e. $\frac{5}{100}$.

In practice, giving some number of per cents, one means so many hundredths of a quantity given in the same \PMlinkescapetext{sentence} or being clear from the context; for example, we can say that the illiteracy in the world is about 20 per cent -- meaning that 20/100 of the adults of the world cannot read.  If we say that the interest (rate) of a loan is 8 per cent, it means that one must pay interest for the loan 8/100 of the amount of the loan in a year.

If a percentage of a quantity has changed e.g. from 12\% to 15\%, we must not say that it has grown 3\% but that it has grown 3 {\em percentage points}.\\

\textbf{Determination of percentage}

How many percent a number $a$ is of a second number $b$?  The answer, the {\em per cent number} $p$, is obtained from
\begin{align}
p = \frac{a}{b}\cdot 100.
\end{align}
The number $b$ here is called the {\em base value} and $a$ the {\em per cent value}(?).  Essentially, the procedure in (1) may be replaced by converting the ratio $\frac{a}{b}$ to hundredths, which can be done formally by multiplying this ratio by\, $1 = \frac{100}{100} = 100\%$:
$$\frac{a}{b} = \frac{a}{b}\cdot 100\,\%.$$



%%%%%
%%%%%
\end{document}
