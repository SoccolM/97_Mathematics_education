\documentclass[12pt]{article}
\usepackage{pmmeta}
\pmcanonicalname{Conjugationmnemonic}
\pmcreated{2013-03-22 16:00:59}
\pmmodified{2013-03-22 16:00:59}
\pmowner{Wkbj79}{1863}
\pmmodifier{Wkbj79}{1863}
\pmtitle{conjugation (mnemonic)}
\pmrecord{6}{38052}
\pmprivacy{1}
\pmauthor{Wkbj79}{1863}
\pmtype{Definition}
\pmcomment{trigger rebuild}
\pmclassification{msc}{97D40}
\pmclassification{msc}{11R04}
\pmrelated{AlgebraicConjugates}
\pmrelated{Division}
\pmrelated{DifferenceOfSquares}
\pmdefines{rationalize the denominator}
\pmdefines{rationalize the numerator}

% this is the default PlanetMath preamble.  as your knowledge
% of TeX increases, you will probably want to edit this, but
% it should be fine as is for beginners.

% almost certainly you want these
\usepackage{amssymb}
\usepackage{amsmath}
\usepackage{amsfonts}

% used for TeXing text within eps files
%\usepackage{psfrag}
% need this for including graphics (\includegraphics)
%\usepackage{graphicx}
% for neatly defining theorems and propositions
%\usepackage{amsthm}
% making logically defined graphics
%%%\usepackage{xypic}

% there are many more packages, add them here as you need them

% define commands here

\begin{document}
In pre-college mathematics, students typically learn how to rationalize the denominator (or, in some cases, numerator) of expressions such as $\displaystyle \frac{3}{\sqrt{11}+2}$ and $\displaystyle \frac{\sqrt{x+h}-\sqrt{x-h}}{2h}$.  In \PMlinkescapetext{order} to do this, they multiply the numerator and denominator of the fraction by an algebraic conjugate (or, in some cases, its negative) to eliminate the \PMlinkname{square root(s)}{SquareRoot} in the appropriate part of the fraction.  Typically, the only algebraic conjugates that pre-college students encounter are those in some quadratic extension.

Most students who have advanced far enough in mathematics to encounter rationalizing denominators or numerators have also encountered some (usually Indo-European) foreign \PMlinkescapetext{language}.  Such students are familiar with the concept of \PMlinkescapetext{conjugation} of verbs, in which the ending of the verb changes to make agreement with the person and number of the subject.  A helpful mnemonic for students to \PMlinkescapetext{calculate} the algebraic conjugates that they need to use is pointing out to them that the procedure in mathematics is \PMlinkescapetext{similar} (and actually easier) than in foreign \PMlinkescapetext{languages}.  The algebraic conjugates (or their negatives) that they need are nothing more than changing the ending of the number.  For example, the way that a pre-college student is taught to rationalize the denominator of an expression such as $\displaystyle \frac{3}{\sqrt{11}+2}$ is:

\begin{center}
$\begin{array}{rl}
\displaystyle \frac{3}{\sqrt{11}+2} & \displaystyle =\frac{3}{\sqrt{11}+2} \cdot \frac{\sqrt{11}-2}{\sqrt{11}-2} \\
& \\
& \displaystyle =\frac{3\sqrt{11}-6}{11-4} \\
& \\
& \displaystyle =\frac{3\sqrt{11}-6}{7} \end{array}$
\end{center}
%%%%%
%%%%%
\end{document}
