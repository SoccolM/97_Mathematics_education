\documentclass[12pt]{article}
\usepackage{pmmeta}
\pmcanonicalname{FactorsWithMinusSign}
\pmcreated{2015-02-04 12:30:11}
\pmmodified{2015-02-04 12:30:11}
\pmowner{pahio}{2872}
\pmmodifier{pahio}{2872}
\pmtitle{factors with minus sign}
\pmrecord{7}{40028}
\pmprivacy{1}
\pmauthor{pahio}{2872}
\pmtype{Topic}
\pmcomment{trigger rebuild}
\pmclassification{msc}{97D40}
\pmclassification{msc}{13A99}
\pmsynonym{sign rules for products}{FactorsWithMinusSign}
%\pmkeywords{product}
%\pmkeywords{power}
\pmrelated{GeneralAssociativity}
\pmrelated{Multiplication}
\pmrelated{DoublyEvenNumber}

% this is the default PlanetMath preamble.  as your knowledge
% of TeX increases, you will probably want to edit this, but
% it should be fine as is for beginners.

% almost certainly you want these
\usepackage{amssymb}
\usepackage{amsmath}
\usepackage{amsfonts}

% used for TeXing text within eps files
%\usepackage{psfrag}
% need this for including graphics (\includegraphics)
%\usepackage{graphicx}
% for neatly defining theorems and propositions
 \usepackage{amsthm}
% making logically defined graphics
%%%\usepackage{xypic}

% there are many more packages, add them here as you need them

% define commands here

\theoremstyle{definition}
\newtheorem*{thmplain}{Theorem}

\begin{document}
\PMlinkescapeword{factor} \PMlinkescapeword{factors} \PMlinkescapeword{base} 
\PMlinkescapeword{power} \PMlinkescapeword{powers}

The sign (cf. plus sign, opposite number) rule
\begin{align}
(+a)(-b) = -(ab),
\end{align}
derived in the 
\PMlinkname{parent entry}{productofnegativenumbers} 
and concerning numbers and elements $a$, $b$ of an arbitrary 
ring, may be generalised to the following

\textbf{Theorem.}\, If the sign of one \PMlinkname{factor}{Product} in a ring product is changed, the sign of the product changes.

\textbf{Corollary 1.}\, The product of real numbers is equal to the product of their absolute values equipped with the ``$-$'' sign if the number of negative factors is odd and with ``$+$'' sign if it is even.  Especially, any odd power of a negative real number is negative and any even power of it is positive.

\textbf{Corollary 2.}\, Let us consider natural powers of a ring element.  If one changes the sign of the base, then an odd power changes its sign but an even power remains unchanged:
$$(-a)^{2n+1} = -a^{2n+1}, \quad (-a)^{2n} = a^{2n} \qquad (n \in \mathbb{N})$$

%%%%%
%%%%%
\end{document}
